\documentclass[14pt, a4paper]{article}
\usepackage{minitoc}
\usepackage[left=3.00cm, right=2.5cm, top=2.00cm, bottom=2.00cm]{geometry}
\usepackage{amsmath}
\usepackage{amssymb}
\usepackage{amsthm}
\usepackage{thmtools}
\usepackage{mathtools}
\usepackage{graphicx}
%\usepackage{algpseudocode}
%\usepackage{algorithm}
\usepackage[ruled,vlined,linesnumbered,algosection]{algorithm2e}
\usepackage{blindtext}
\usepackage{setspace}
\usepackage[utf8]{inputenc}
\usepackage[utf8]{vietnam}
\usepackage[center]{caption}
\usepackage[shortlabels]{enumitem}
\usepackage{fancyhdr} % header, footer
\usepackage{hyperref} % loại bỏ border với mục lục và công thức
\usepackage[nonumberlist, nopostdot, nogroupskip]{glossaries}
\usepackage{glossary-superragged}
\usepackage{tikz,tkz-tab}
\setglossarystyle{superraggedheaderborder}
\pagestyle{fancy}
%\usepackage[style=numeric,sortcites]{biblatex}
%\addbibresource{ref.bib}
%\usepackage[numbers]{natbib}
\usepackage{indentfirst}
\usepackage{multirow}
\usepackage[natbib,backend=biber,style=ieee, sorting=ynt]{biblatex}
\usepackage{cancel}
\bibliography{ref.bib}

\graphicspath{{./figures/}}


\hypersetup{
    colorlinks=false,
    pdfborder={0 0 0},
}


\fancyhf{}
\rhead{\textbf{Môn học: Các phương pháp ngẫu nhiên và ứng dụng}}
\lhead{\textbf{GVHD: PGS. TS. Tạ Công Sơn}}
\rfoot{\thepage}
\lfoot{\textbf{Học viên thực hiện: Nguyễn Chí Thanh - 21007925}}
\renewcommand{\headrulewidth}{0.4pt}
\renewcommand{\footrulewidth}{0.4pt}


\numberwithin{equation}{section}
\numberwithin{figure}{section}

\setlength{\parindent}{0.5cm}

\setcounter{secnumdepth}{3} % Cho phép subsubsection trong report
\setcounter{tocdepth}{3} % Chèn subsubsection vào bảng mục lục

\newtheorem{dl}{Định lý}
\newtheoremstyle{sltheorem}
{}                % Space above
{}                % Space below
{\normalfont}        % Theorem body font % (default is "\upshape")
{}                % Indent amount
{\bfseries}       % Theorem head font % (default is \mdseries)
{.}               % Punctuation after theorem head % default: no punctuation
{ }               % Space after theorem head
{}                % Theorem head spec
\theoremstyle{sltheorem}
\newtheorem{vd}{Ví dụ}
\newtheoremstyle{soltheorem}
{}                % Space above
{}                % Space below
{\normalfont}        % Theorem body font % (default is "\upshape")
{}                % Indent amount
{\bfseries}       % Theorem head font % (default is \mdseries)
{.}               % Punctuation after theorem head % default: no punctuation
{\newline}               % Space after theorem head
{}                % Theorem head spec
\theoremstyle{soltheorem}
\newtheorem*{loigiai}{Lời giải}

\numberwithin{dl}{section}
\numberwithin{vd}{section}

\doublespacing

\begin{document}
    \begin{titlepage}

        \newcommand{\HRule}{\rule{\linewidth}{0.5mm}} % Defines a new command for the horizontal lines, change thickness here

        \center % Center everything on the page

        %----------------------------------------------------------------------------------------
        %	HEADING SECTIONS
        %----------------------------------------------------------------------------------------
        \textsc{\LARGE Đại học Quốc Gia Hà Nội}\\[0.5cm]
        \textsc{\LARGE Trường đại học Khoa học tự nhiên}\\[0.5cm] % Name of your university/college
        \textsc{\LARGE Khoa Toán - Cơ - Tin học}\\[0.5cm]

        \includegraphics[scale=0.2]{HUS-logo.jpg}\\[0.5cm]

        \textsc{\Large Chuyên ngành: Khoa học dữ liệu}\\[0.5cm] % Major heading such as course name


        %----------------------------------------------------------------------------------------
        %	TITLE SECTION
        %----------------------------------------------------------------------------------------

        \HRule \\[0.4cm]
        { \huge \bfseries TIỂU LUẬN MÔN HỌC}\\[0.4cm] % Title of your document
        \HRule \\[1.5cm]

        \textsc{\Large Môn học: Các phương pháp ngẫu nhiên và ứng dụng}\\[1cm] % Minor heading such as course title


        \textsc{\Large Đề tài: Xích Markov và Lý thuyết phân nhánh}\\[2cm]


        %----------------------------------------------------------------------------------------
        %	AUTHOR SECTION
        %----------------------------------------------------------------------------------------
        \begin{minipage}{0.4\textwidth}
            \begin{flushleft} \large
            \emph{Giảng viên hướng dẫn:} \\
            PGS. TS. Tạ Công Sơn % Supervisor's Name
            \end{flushleft}
        \end{minipage}\\[0.5cm]

        \begin{minipage}{0.4\textwidth}
        \begin{flushleft} \large
        \emph{Học viên thực hiện:}\\
        Nguyễn Chí Thanh \\
        MSHV: 21007925 \\ % Your name
        Lớp: Khoa học dữ liệu - K4
        \end{flushleft}
        \end{minipage}


        % If you don't want a supervisor, uncomment the two lines below and remove the section above
        %\Large \emph{Author:}\\
        %John \textsc{Smith}\\[3cm] % Your name

        %----------------------------------------------------------------------------------------
        %	DATE SECTION
        %----------------------------------------------------------------------------------------

        % I don't want day because it is English
        % {\large \today}\\[2cm] % Date, change the \today to a set date if you want to be precise

        %----------------------------------------------------------------------------------------
        %	LOGO SECTION
        %----------------------------------------------------------------------------------------

        %\includegraphics{logo/rsz_3logo-khtn.png}\\[1cm] % Include a department/university logo - this will require the graphicx package

        %----------------------------------------------------------------------------------------

        \vfill % Fill the rest of the page with whitespace

    \end{titlepage}

    \cleardoublepage
    \pagenumbering{gobble}
    \tableofcontents
    \newpage
    \listoffigures
    \newpage
    \glsaddall 
    \renewcommand*{\glossaryname}{Danh mục các từ viết tắt}
    \renewcommand*{\acronymname}{Danh sách từ viết tắt}
    \renewcommand*{\entryname}{Viết tắt}
    \renewcommand*{\descriptionname}{Viết đầy đủ}
    \printnoidxglossary
    \cleardoublepage
    \pagenumbering{arabic}

    %\maketitle

    \newpage

    \nocite{*}

    \begin{center}
    \section*{LỜI MỞ ĐẦU}
    \end{center}
    \addcontentsline{toc}{section}{{\bf LỜI MỞ ĐẦU}\rm}

    \newpage

    \section{Số bước trung bình chuyển tiếp giữa các trạng thái tạm thời}

    Ta xét một xích Markov trạng thái hữu hạn và giả định rằng các trạng thái được đánh số $T=\lbrace 1, 2, \dots, t \rbrace$ ký hiệu tập các trạng thái tạm thời.
    Ta đặt:

    \begin{equation*}
        \mathbf{P}_T = \begin{bmatrix} P_{11} & P_{12} & \dots & P_{1t} \\ \vdots & \vdots & \vdots & \vdots \\ P_{t1} & P_{t2} & \dots & P_{tt}  \end{bmatrix}
    \end{equation*}

    Và ta cần chú ý rằng ma trận $\mathbf{P}_T$ chỉ xác định xác suất chuyển tiếp từ các trạng thái tạm thời sang trạng thái tạm thời, và tổng của một số hàng nhỏ hơn 1 (mặt khác $T$ có thể là một tập đóng của các trạng thái).
    
    Ta xét hai trạng thái tạm thời $i$ và $j$, ta đặt $s_{ij}$ ký hiệu là số bước kỳ vọng mà xích Markov đến trạng thái $j$, khi ta biết xích Markov bắt đầu ở trạng thái $i$.
    Đặt $\delta_{i,j}=1$ khi $i=j$ và bằng 0 nếu ngược lại.
    Ta tính $s_{ij}$ sử dụng công thức kỳ vọng đầy đủ:

    \begin{equation} \label{eq:expected_number_periods_i_to_j}
        \begin{aligned}
            s_{ij} &= \delta_{i,j} + \sum_{k \in T} P_{ik}s_{kj} \\
            &= \delta_{i,j} + \sum_{k=1}^t P_{ik} s_{kj}
        \end{aligned}
    \end{equation}

    với đẳng thức cuối cùng của công thức trên cho ta thấy không thể chuyển từ một trạng thái hồi quy sang một trạng thái tạm thời, ý nghĩa rằng $s_{kj}=0$ khi $k$ là một trạng thái hồi quy.

    Ta đặt $\mathbf{S}$ ký hiệu là ma trận với các giá trị $s_{ij}, i, j = 1, \dots, t$. Ta có ma trận $\mathbf{S}$:

    \begin{equation*}
        \mathbf{S} = \begin{bmatrix} s_{11} & s_{12} & \dots & s_{1t} \\ \vdots & \vdots & \vdots & \vdots \\ s_{t1} & s_{t2} & \dots & s_{tt}  \end{bmatrix}
    \end{equation*}

    Ta viết lại công thức \ref{eq:expected_number_periods_i_to_j} dưới dạng ma trận:

    \begin{equation*}
        \mathbf{S} = \mathbf{I} + \mathbf{P}_T \mathbf{S}
    \end{equation*}

    với $\mathbf{I}$ là ma trận đơn vị kích thước $t$. Bởi vì công thức trên tương đương với:

    \begin{equation*}
        (\mathbf{I} - \mathbf{P}_T) \mathbf{S} = \mathbf{I}
    \end{equation*}

    Bằng cách nhân cả hai vế với $(\mathbf{I} - \mathbb{P}_T)^{-1}$ ta thu được:

    \begin{equation*}
        \mathbf{S} = (\mathbf{I} - \mathbf{P}_T)^{-1}
    \end{equation*}

    Với mỗi đại lượng $s_{ij}, i \in T, j \in T$, ta có thể tính đại lượng này bằng cách lấy phần tử tại hàng $i$ và cột $j$ của ma trận nghịch đảo của ma trận $\mathbf{I} - \mathbf{P}_T$ (sự tồn tại của ma trận nghịch đảo này dễ dàng chứn minh được).

    \begin{vd} \label{vd:gambler-ruin-problem}
        Ta xét bài toán người đánh bạc với $p=0.4$ và $N=7$. Bắt đầu với 3 đô la, xác định:
        \begin{enumerate}[label=(\alph*)]
            \item Số ván kỳ vọng để người chơi bài có 5 đô la.
            \item Số ván kỳ vọng để người chơi bài có 2 đô la.
        \end{enumerate}
    \end{vd}

    \begin{loigiai}
        Ma trận $\mathbf{P}_T$ xác định các giá trị $P_{ij}, i, j \in \lbrace 1, 2, 3, 4, 5, 6 \rbrace$ được xác định như sau:

        \begin{equation*}
            \mathbf{P}_T = \begin{bmatrix}
                0 & 0.4 & 0 & 0 & 0 & 0 \\
                0.6 & 0 & 0.4 & 0 & 0 & 0 \\
                0 & 0.6 & 0 & 0.4 & 0 & 0 \\
                0 & 0 & 0.6 & 0 & 0.4 & 0 \\
                0 & 0 & 0 & 0.6 & 0 & 0.4 \\
                0 & 0 & 0 & 0 & 0.6 & 0 \\
            \end{bmatrix}
        \end{equation*}

        Ta tính nghịch đảo ma trận $\mathbf{I} - \mathbf{P}_T$, ta thu được:

        \begin{equation*}
            \mathbf{S}=(\mathbf{I}-\mathbf{P}_T)^{-1}=
            \begin{bmatrix}
            1.6149 & 1.0248 & 0.6314 & 0.3691 & 0.1943 & 0.0777 \\
            1.5372 & 2.5619 & 1.5784 & 0.9228 & 0.4857 & 0.1943 \\
            1.4206 & 2.3677 & 2.9990 & 1.7533 & 0.9228 & 0.3691 \\
            1.2458 & 2.0763 & 2.6299 & 2.9990 & 1.5784 & 0.6314 \\
            0.9835 & 1.6391 & 2.0763 & 2.3677 & 2.5619 & 1.0248 \\
            0.5901 & 0.9835 & 1.2458 & 1.4206 & 1.5372 & 1.6149
            \end{bmatrix}
        \end{equation*}

        Vì vậy ta thu được $s_{3,5}=0.9228$ (ván) và $s_{3,2}=2.3677$ (ván).
    \end{loigiai}

    Với $i \in T, j \in T$, đại lượng $f_{ij}$ chính là xác suất để để xích Markov tạo ra một quỹ đạo chuyển tiếp đạt đến trạng thái $j$ khi biết xích Markov bắt đầu từ trạng thái $i$ dễ dàng được xác định từ $\mathbf{P}_T$.
    Để xác định mối liên hệ, ta bắt đầu từ biệc sử dụng công thức tính $s_{ij}$ dựa trên hệ đầy đủ là xích Markov đã từng đi qua trạng thái $j$ hay chưa.
    Sử dụng công thức kỳ vọng đầy đủ ta thu được:

    \begin{equation*}
        \begin{aligned}
            s_{ij} &= E \lbrack \text{ số bước để đến được } j \vert \text{ bắt đầu tại } i, \text{ đã từng đi qua } j \rbrack f_{ij} \\ & + E \lbrack \text{ số bước để đến được } j \vert \text{ bắt đầu tại } i, \text{ chưa từng đi qua } j \rbrack (1 - f_{ij}) \\
            &= (\delta_{i, j} + s_{{jj}}) f_{ij} + \delta_{ij}(1 - f_{ij}) \\
            &= \delta_{i,j} + f_{ij} s_{jj}
        \end{aligned}
    \end{equation*}

    với $s_{jj}$ là số bước kỳ vọng cần thêm vào khi trạng thái $j$ đã đạt và tiếp tục một lần nữa quay lại trạng thái $j$.
    Giải phương trình trên ta thu được:

    \begin{equation*}
        f_{ij} = \dfrac{s_{ij} - \delta_{i,j}}{s_{jj}}
    \end{equation*}

    \begin{vd}
        Trong ví dụ \ref{vd:gambler-ruin-problem}, xác suất là bao nhiêu để người chơi có lúc có 1 đô la?
    \end{vd}

    \begin{loigiai}
        Với $s_{31}=1.4206$ và $s_{11}=1.6149$, ta thu được:
        
        \begin{equation*}
            f_{31} = \dfrac{s_{31}}{s_{11}}=0.8797
        \end{equation*}

        Để kiểm tra lại, ta chú ý rằng $f_{31}$ chính là xác suất người đánh bạc bắt đầu với 3 đô la và có 1 đô la trước khi đạt được 7 đô la.
        Đây chính là xác suất để người đánh bạc mất 2 đô la trước khi có thêm 4 đô la cũng bằng xác suất người đánh bạc bắt đầu với 2 đô la và thu cuộc trước khi đạt được 6 đô la.
        Vì vậy:

        \begin{equation*}
            f_{31} = - \dfrac{1 - (0.6/0.4)^2}{1-(0.6/0.4)^6}=0.8797
        \end{equation*}
        và kết quả trùng khớp với cách làm đầu tiên.

        Giả sử ta đang quan tâm đến số bước kỳ vọng để xích Markov đi đến một tập các trạng thái $A$, không nhất thiết phải là các trạng thái lặp lại.
        Ta có thể giả định về tình huống trước bằng cách làm cho tất cả các trạng thái trong $A$ là trạng thái hấp thụ bằng cách đặt lại xác suất chuyển tiếp của các trạng thái trong tập $A$ thỏa mãn:

        \begin{equation*}
            P_{ii} = 1, i \in A
        \end{equation*}

        Việc này biến các trạng thái trong tập $A$ thành trạng thái hồi quy, và biến đổi bất kỳ trạng thái nào ngoài tập $A$ nhưng cuối cùng cũng chuyển tiếp vào một trạng thái trong tập $A$ thành một trạng thái tạm thái.
        Vì vậy, cách tiếp cận trước đây của ta có thể được sử dụng

        \section{Các quá trình phân nhánh}

        Trong phần này ta sẽ xem xét một lớp các xích Markov được gọi là các \textit{quá trình phân nhánh}.
        Lớp xích Markov này có ứng dụng rất rộng rãi trong y học, xã hội học và khoa học kỹ thuật.

        Ta xét một quần thể bao gồm các cá thể có thể tạo ra các thế hệ tiếp theo cùng loài.
        Ta giả định rằng từng cá thể đến cuối thời gian sống tạo ra $j$ con với xác suất $P_j, j \geq 0$ và độc lập với số con được sinh ra bởi cá thể khác.
        Ta giả sử $P_j < 1 \forall j \geq 0$.
        Số cá thể ban đầu được ký hiệu là $X_0$, được gọi là kích cỡ của thế hệ thứ 0.
        
    \end{loigiai}
\end{document}